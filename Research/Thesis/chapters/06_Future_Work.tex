\chapter{Future Work}
As part of our research we extensively analyzed many aspects of different implementations that solve our use case. However, there is still room for improvement and there were also some new questions raised. We would like to point out some of the most relevant questions and areas for future research in the following section.\\
The \gls{cnn} used for our implementation is based on the \gls{mobilenet} neural network originally designed for general object recognition tasks. Therefore it contains many additional layers with little relevance for face detection. These layers pose a significant overhead which we believe can be optimized with a more specific neural network design.\\
There is also a lot of room for improving the bounding boxes of the object tracker. We currently use the KCF object tracker to track the entire face. This results in a significant overhead compared to the MOSSE tracker and in some cases the facial landmarks at the edge of a face cannot be detected as the bounding boxes do not scale correctly. We believe it would make more sense to track only specific features of the face (for instance both eyes) and then use the distance and tilt between the tracked eyes to scale the bounding boxes accordingly.\\
The only component of our system that was not optimized is the emotion classifier as the existing implementation was accurate and fast enough for our use case. For future development it would make sense to replace the current implementation based on a support vector machine with a neural network to improve accuracy. A neural network could also be trained to recognize not just emotions but also people to enable even more use cases such as face verification.\\
Lastly testing our implementation on different devices presents another interesting area of research as each embedded device comes with its own characteristics and computational capabilities (like the recently released Raspberry PI 4). 
%Future Work: In science folklore, the merit of a research question is compounded by the number of interesting follow-up research questions it raises. So to show the merit of the problem you worked on, you list these questions here. If you don’t care about research folklore (I did not as a student), this chapter is still useful: whenever you stumble across something that you should do if you had unlimited time, but cannot do since you don’t, you describe it here. Typical candidates are evaluation on more study objects, investigation of potential threats to validity, … The point here is to inform the reader (and your supervisor) that you were aware of these limitatons. Limit this chapter to very few pages. Two is entirely fine, even for a Master’s thesis.